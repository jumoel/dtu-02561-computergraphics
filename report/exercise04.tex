
\chapter{Ugeopgave 4}
\label{cha:ugeopgave-4}

Form�let med denne opgave er at blive fortrolig med input og
vinduesh�ndtering i Glut og OpenGL.

\section{Del 1: \texttt{newpaint}}
\label{sec:del-1:-newpaint}

\texttt{Newpaint} har fem knapper, der hver har sin egen
funktionalitet. Knappernes funktionalitet aktiveres ved at trykke p�
knapperne.

\begin{itemize}
\item Tegning af en linje, ved at venstreklikke p� startpunktet og
  efterf�lgende slutpunktet.
\item Tegning af en firkant, ved at venstreklikke p� to hj�rnepunkter.
\item Tegning af en trekant, ved at venstreklikke p� tre hj�rnepunkter.
\item Tegning af et punkt, ved at venstreklikke.
\item Tegning af tekst, ved at venstreklikke p� startpositionen,
  skrive en tekst med tastaturet og venstreklikke et sted i programmet
  for at afslutte indtastningen.
\end{itemize}

Endvidere er der popup-menuer, der aktiveres ved h�jre- eller
midterklik p� musen. H�jrekliksmenuen lader en afslutte programmet
(``quit'') eller rydde tegnefladen (``clear''). Midterkliksmenuen
lader en s�tte tegnefarven, for�ge eller formindske pixelst�rrelsen,
samt aktivere eller dekativere udfyldning af figurerne.

\texttt{pick}-funktionen bliver brugt til at bestemme om der er
trykket p� en knap, og hvis det er tilf�ldet hvilken knap der er
blevet trykket p�. Det g�r den ved at unders�ge om musetrykket er
inden for rammerne af knapperne. Da hver knap altid fylder 10\% af
sk�rmbredden er dette tjek nemt.
